\documentclass{report}
\usepackage[utf8]{inputenc}
\usepackage{graphics}
\graphicspath{ {./images/} }
\usepackage{multicol}

% width,height
\usepackage[a4paper, total={7in, 10in}]{geometry}



\begin{document}

    
    \begin{center}
        \section*{Big Sensor Data Systems for Smart Cities\\AUTHORS:Li-Minn Ang Senior Member, IEEE, Kah Phooi Seng Member, IEEE, Adamu Murtala Zungeru
Member, IEEE, and Gerald Ijemaru, Member, IEEE}
    
    \setlength{\columnsep}{1.0cm}
   
    Shruthi.M\\
    AIDS-B\\
    211101124
    \end{center}
    \begin{center}
        \section*{}
    \end{center}
    
\setlength{\columnsep}{1.0cm}
    \large
    \section{Summary}
   
    
    This research paper defines big sensor data,its application. The smart cities layer challenges are explained. The key aim is to improve solutions in the deployment of smart cities. The emerging field of networked sensing technologies with provides big sensor data. The People/Creature Domain is an important part of the big sensor data framework. The smart cities framework consisting of 5 layers which provides a logical flow is detailed. The constraints due to power consumption is detailed. The barriers faced by standard Big data systems are compared to their deployment in smart cities. The usage of cloud storage for smart cities ecosystem and the challenges faced is discussed. The machine learning and data mining for statistical data analysis is explained. The increasing demand for data storage with all time availability and varying level of consistency are few challenges data management. The 2 types of sensor-based applications are functional and recoverable. The 4 challenges for designing big sensor data systems at the Analytics Layer: (1) Cross-domain machine learning; (2) Data inference; (3) Real-time applications; and (4) New uses of sensing infrastructures. Many real-time and real-world applications of smart cities are illustrated.
    
\begin{multicols}{1}    
    \section*{Key contribution/ideas from the author}
   
    Big data techniques are targeted towards solving system level problems that cannot be solved by conventional methods and technologies. 'People as sensors' is an important part of the big sensor data framework. Sensor/Things Domain includes technologies like wireless sensor networks (WSNs) and the Internet-of-Things (IoT). The boundary for the IoT does not include the People/Creature Domain.\\
Smart city applications would generate huge amounts of data from a wide variety of sources including sensors, mobile phones, and people’s social networks. Research challenges remain on how to integrate and utilize this data. The five layers of the smart city framework provide a logical flow that enables the stakeholders in the smart cities to view the flow of information. This paper aims to get insights for using big sensor data systems for deployment in smart cities.\\
The connectivity and communication of systems and/or platforms in most cases require the IoT. The connectivity and communication of systems and/or platforms in most cases require the IoT. Some of these techniques include: Low Power Wide Area Networks (LPWAN),Cellular technologies, Wireless Personal Area Networks (WPAN). e the wireless network of the sensor devices will have to rely on devices in the smart city to deliver information in a multi-hop manner, communication protocols should be robust to device failures and prevent single point-to-point situations, for information not to be lost in case a sensor dies.\\
Some of the interoperability issues in big sensor systems involve connection admission control, specification of protocol suites, end-to-end quality of service, provision of basic and enhanced services, user selection of transit networks and content providers, user data element format, processing, and retrieval and storage of information. Public cloud storage infrastructure usually consists of low-cost storage nodes with directly attached commodity drives with an object-based storage stack that manages the distribution of content across nodes. Private storage clouds are usually for a single tenant, even though larger enterprises may use multi-tenancy features to segregate access by departments or office locations. Hybrid Cloud Storage: Users of this service manage resources both externally and in-house.\\
Big sensor data management for smart cities is dependent upon the following key factor and challenges: data privacy, data security, governance and ethical size of big data structures, legal and technological implications, and policy issues. Useful approaches for data cleaning for sensor data have used techniques like spatiotemporal regression and Kalman filters. The four challenges for designing big sensor data systems at the Analytics Layer: (1) Cross-domain machine learning; (2) Data inference; (3) Real-time applications; and (4) New uses of sensing infrastructures. For each of the N data elements, there is a corresponding geographic location which was obtained from a separate set of land use and geographic data at the tax lot level.\\

    

    \section*{My views}
   
    This paper details about the usage of big sensor data for building smart cities. Smart cities are of great utility. Smart cities are aimed at increasing operational efficiency, share information with the public and improve both the quality of government services and citizen welfare. The insights of challenges faced in designing the layers of smart cities and its possible solutions are detailed which helps in improving the deployment of smart cities in future.
    
    
    \section*{Agreement, Pitfalls and Fallacies}
    Although, many techniques for handling a single form of sensor data type have been well established, what is still missing are approaches and models for handling and integrating the diverse and historical sources of cross-domain and multi-modal spatial-temporal data which characterizes big sensor data systems to produce valuable outcomes for human society. The convergence of the three domains in the big sensor framework, and the emergence of the IoP as well as the current IoT enables new sensing applications to be designed and developed.
    
    
\end{multicols}
\end{document}